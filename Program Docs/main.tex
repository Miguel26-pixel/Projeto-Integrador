\documentclass[a4paper, 11pt]{report}

%%%%%%%%%%%%
% Packages %
%%%%%%%%%%%%

\usepackage[english]{babel}
\usepackage[noheader]{packages/sleek}
\usepackage{packages/sleek-title}
\usepackage{packages/sleek-theorems}
\usepackage{packages/sleek-listings}
\usepackage{datetime}
\usepackage{graphicx}
\usepackage{multirow}	% for multirow
\usepackage{algorithm}
\usepackage{algpseudocode}
\RequirePackage{listings}
\usepackage[most]{tcolorbox}
\usepackage[table]{colortbl}%

\usepackage{etoolbox}
\makeatletter
% no new page for \chapter
\patchcmd{\chapter}{\if@openright\cleardoublepage\else\clearpage\fi}{}{}{}

\usepackage{sectsty}
\usepackage{titlesec}


\titleformat{\chapter}[display]
  {\normalfont\bfseries}{}{0pt}{\Large}
\titlespacing*{\chapter}{0pt}{2pt}{2pt}

\chaptertitlefont{\LARGE}
\sectionfont{\large}

%%%%%%%%%%%%%%
% Title-page %
%%%%%%%%%%%%%%

\logo{logo/logo.png}
\institute{Faculdade de Engenharia}
\faculty{Universidade do Porto}

\title{Ubiquitous Sensing - Sensing, Storing and Processing Plant Sensor Data}
\subtitle{Capstone Project, Project Plan}
\author{Mário \textsc{Travassos} up201905871@edu.fe.up.pt \\ Miguel \textsc{Amorim} up201907756@edu.fe.up.pt \\ Nuno \textsc{Alves} up201908250@edu.fe.up.pt \\ Ricardo \textsc{Ferreira} up201907835@edu.fe.up.pt}

\date{\formatdate{27}{03}{2022}}

%%%%%%%%%%
% Others %
%%%%%%%%%%s

\lstdefinestyle{latex}{
    language=TeX,
    style=default,
    %%%%%
    commentstyle=\ForestGreen,
    keywordstyle=\TrueBlue,
    stringstyle=\VeronicaPurple,
    emphstyle=\TrueBlue,
    %%%%%
    emph={LaTeX, usepackage, textit, textbf, textsc}
}

\FrameTBStyle{latex}

\def\tbs{\textbackslash}

%%%%%%%%%%%%
% Document %
%%%%%%%%%%%%
\begin{document}
    \maketitle
    \romantableofcontents
    
\chapter{Summary}

Plants present a unique opportunity for a sustainable sensing platform and they include various biological mechanisms that are controlled by environmental variables, such as light, gas concentration and are even sensitive to touch. 
In this project, we will explore the ability to observe, acquire and process, in a controlled environment (a lab and/or a greenhouse facility) electrical signals generated by several plants using specific bio-electrodes being currently developed in a laboratory setting. The candidates will assemble a portable (ideally embedded) data acquisition and processing hardware kit (based on an existent hardware platform consisting of a CPU and interface with sensors - such as a Raspberry Pi) and the corresponding basic signal processing to acquire, pre-process and store on a computer server digital representations of the electrical responses observed with the plants under study alongside environmental conditions (such as temperature, light, humidity). The developed data collection solution will also include the facility to stream data to a desktop or cloud server and the development of a software dashboard accessible via a web browser to show the values of the sensors in real-time.


\break

\chapter{Objectives}

The expected project outcomes will consist of a hardware/software data acquisition, data storage and processing system designed to interact with selected plant organisms in a laboratory setting. In addition, the project will develop a dashboard to show via web the real-time sensing data.

This project will include a collaboration with the Biophysics and Bioelectronics Lab. of the University of Coimbra where the final stages of data acquisition and field experiments will be carried out once the developed solution has been tested and validated using plant proxies in a laboratory setting at FEUP.


\break

\chapter{Project Schedule}

% cool gantt chart graphic

\begin{table}[h!]
    \centering
    \begin{tabular}{|c|l||c|p{0.25cm}|p{0.25cm}|p{0.25cm}|p{0.25cm}|p{0.25cm}|p{0.25cm}|p{0.25cm}|p{0.25cm}|c|}
     \hline
     \multicolumn{2}{|c||}{\textbf{Task}} &  \textbf{March} & \multicolumn{4}{c|}{\textbf{April}} & \multicolumn{4}{c|}{\textbf{May}} & \textbf{June} \\ 
     \hline\hline
     &1. {Research Sensors and Compatibility} & \cellcolor{blue!25} &  &  &  &  &  &  &  &  &   \\ 
     \cline{2-12}
     &2. {Connect Raspberry PI to ADC}  &  & \cellcolor{blue!25} & \cellcolor{blue!25} &  &  &  &  &  &  &   \\ 
     \cline{2-12}
     &3. {Connect Raspberry PI to Sensors}  &  & \cellcolor{blue!25} & \cellcolor{blue!25} &  &  &  &  &  &  & \\
     \cline{2-12}
     \multirow{-4}{*}{\rotatebox[origin=c]{90}{RPI}}&4. {Connect Raspberry PI to WiFi}  &  & \cellcolor{blue!25} & \cellcolor{blue!25} &  &  &  &  &  &  &  \\
     \hline
     &5. {Configure Server and Database}  &  & \cellcolor{blue!25} & \cellcolor{blue!25} & \cellcolor{blue!25} & \cellcolor{blue!25} &  &  &  &  &  \\
     \cline{2-12}
     &6. {Create Database}  &  & \cellcolor{blue!25} & \cellcolor{blue!25} & \cellcolor{blue!25} & \cellcolor{blue!25} &  &  &  &  &  \\
     \cline{2-12}
     \multirow{-3}{*}{\rotatebox[origin=c]{90}{DB}}&7. {Send Sensor Data to Database}  &  & \cellcolor{blue!25} & \cellcolor{blue!25} & \cellcolor{blue!25} & \cellcolor{blue!25} &  &  &  &  &  \\
     \hline
     &8. {UI Mockups}  &  & \cellcolor{blue!25} & \cellcolor{blue!25} &  &  &  &  &  &  &  \\
    \cline{2-12}
     &9. {Front-end UI}  &  & \cellcolor{blue!25} & \cellcolor{blue!25} &  &  &  &  &  &  &  \\
     \cline{2-12}
     & 10. {Real-Time Graphs}   &  & \cellcolor{blue!25} & \cellcolor{blue!25} & \cellcolor{blue!25} & \cellcolor{blue!25} &  &  &  &  &  \\
    \cline{2-12}
     \multirow{-4}{*}{\rotatebox[origin=c]{90}{DASH}} & 11. {Multiple Plant Configuration}  &  &  &  &  &  & \cellcolor{blue!25} &  &  &  &  \\
     \hline
    \end{tabular}
\end{table}

\section{Milestone - Raspberry PI [RPI] }

This first milestone's objective is to understand how the Raspberry PI works and how the team can connect sensors to it. By the end of this milestone, the team has assembled a Raspberry PI that collects various real-world information like ambient light, humidity, and temperature, as well as electrical signals generated by a plant.


\subsection{Research Sensors and Compatibility}

This task has the objective of research for the sensors of light, humidity, ultrasound and temperature that can be use with the version of Raspberry PI.

\subsection{Connect Raspberry PI to ADC}

This task has the objective of understanding how the Raspberry PI connects to ADC (Analog Data Converter), since it's with that connection that the Raspberry PI will receive the packages with data collected from the plants.

\subsection{Connect Raspberry PI to Sensors}

This task has the objective of understanding how the Raspberry PI connects to the sensors we got from the research, since the data collected with those sensors is the base to develop and test the project.

\subsection{Connect Raspberry PI to WiFi}

This task has the objective of understanding how the Raspberry PI can be connected to WiFi using a dongle WiFi, so the project will be accessible anywhere through the Internet.

\section{Milestone - Database [DB] }

This second milestone's objective is to configure the server and the database that will receive the data to show in dashboards, in real-time. By the end of this milestone, the team has assembled a server and a database capable of saving data from sensors or the ADC.

\subsection{Configure Server and Database}

This task has the objective of configuring the server and the database that will be saving the data in real-time from the sensors or ADC.

\subsection{Create Database}

This task has the objective of creating the database that will be saving the data in real-time from the sensors or ADC.

\subsection{Send Sensor Data to Database}

This task has the objective of understanding how and what will be send to the database from the data we receive from sensors or from the ADC

\section{Milestone - Online Dashboard [DASH]}

This third milestone's objective is to make a interactive web application where the user can see dashboards with the real-time data collected from the sensors or ADC.

\subsection{UI Mockups}

This task has the objective of making Mockups that show how the user will interact with the application as well as the design.

\subsection{Front-end UI}

This task has the objective of making Mockups that show how the application will look like, the design of the graphs as well.

\subsection{Real-Time Graphs}

This task has the objective of making the real-time graphs that show the data that is being collected in real-time by the sensors on a plant.

\subsection{Multiple Plant Configuration}

This task has the objective of configuring the application in a way that could be data coming from more than on plant at the same time.

\section{Stretch Goals}

The Stretch Goals are objectives the team would like to achieve after all basic functionality is implemented.

\subsection{Plant 3D Render}

This extra goal has the objective of doing an interactive plant 3D render where the user could see the dashboards with the real-time data collected.

\subsection{Infer Meaning of Eletrical Signals using Machine Learning}

This extra goal has the objective of using Machine Learning to Infer Meaning of electrical signals, that way the user could interact with the plant and the application would predict some results.


\end{document}
